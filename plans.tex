% Created 2025-02-12 Wed 14:38
% Intended LaTeX compiler: pdflatex
\documentclass[11pt]{article}
\usepackage[utf8]{inputenc}
\usepackage[T1]{fontenc}
\usepackage{graphicx}
\usepackage{longtable}
\usepackage{wrapfig}
\usepackage{rotating}
\usepackage[normalem]{ulem}
\usepackage{amsmath}
\usepackage{amssymb}
\usepackage{capt-of}
\usepackage{hyperref}
\author{fcalado}
\date{\today}
\title{}
\hypersetup{
 pdfauthor={fcalado},
 pdftitle={},
 pdfkeywords={},
 pdfsubject={},
 pdfcreator={Emacs 29.3 (Org mode 9.6.15)}, 
 pdflang={English}}
\begin{document}

\tableofcontents

\section{week 3: python 2: lists to logic}
\label{sec:org0b39f74}
Agenda
\begin{itemize}
\item review challenge
\item review last week's lessons on Python 1
\item lessons for today: loops, logic
\begin{itemize}
\item incorporate discussion of Butler's chapter
\end{itemize}
\item start working with CSV module
\end{itemize}

\subsection{review}
\label{sec:orga147ad0}
\begin{itemize}
\item data types,
\begin{itemize}
\item \texttt{type()}
\item what are they?
\item how do we check them?
\end{itemize}
\item we save data by creating variables
\item lists
\begin{itemize}
\item list indexing vs slicing
\end{itemize}
\item methods vs functions
\begin{itemize}
\item what are some functions and methods we learned?
\item functions are independant
\item methods depend on objects (also called properties, attributes)
\item can tell in syntax
\end{itemize}
\end{itemize}


\subsection{review challenge}
\label{sec:orgc721252}
Create a list of words from a text, and do things to the list using
list methods.

Removing Gender Ideology and Restoring the EEOC’s Role of Protecting
Women in the Workplace


\subsection{butler}
\label{sec:org2f81e67}
What's her main argument?
\begin{itemize}
\item what's so scary?
\begin{itemize}
\item that people are currently afraid of gender, "the anti-gender
ideology movement" stokes fear.
\item gender ideology is a threat to children, society, the family,
national security, to men and women, heterosexuality.
\end{itemize}
\item the contradictions:
\begin{itemize}
\item the vatican saying that something is a threat to children and the
family, not considering their own harmful history here;
\item how withholding sex education also withholds education around
consent and just how sex works;
\item using the right to gender as a way to take that right away from
others.
\end{itemize}
\end{itemize}

"Phantasm"
\begin{itemize}
\item drawing from psychology, to argue that the fear of gender draws from
real world and pyschic forces, the conscious and unconscious.
\begin{itemize}
\item it becomes a substitute for anxiety about the world.
\end{itemize}
\item fear mongering is a way of getting people to agree, come into your
ranks, subscribe.
\end{itemize}

\begin{quote}
“According to this logic, the anti-gender movement is guided by an
inflammatory syntax: that is, a way of ordering the world that absorbs
and reproduces anxieties and fears about permeability, precarity,
displacement, and replacement; loss of patriarchal power in both the
family and state; and loss of white supremacy and national purity” 22
\end{quote}

Resistance
\begin{itemize}
\item produce a "counter vision".
\end{itemize}

\begin{quote}
“It is up to us to produce a compelling counter-vision, one that would
affirm the rights and freedoms of embodied life that we can, and
should, protect. For in the end, defeating this phantasm is a matter
of affirming how one loves, how one lives in one’s body, the right to
exist in the world without fear of violence or discrimination, to
breathe, to move, to live.” 17
\end{quote}

\begin{quote}
“What form of critical imagination would be powerful enough to oppose
the phantasm? What would it mean to create a form of solidarity and
concerted imagining that would have the power to expose and defeat the
cruel norms and sadistic trends that travel under the name of the
anti–gender ideology movement?” (37).
\end{quote}

\subsection{loops}
\label{sec:orgf0cc56a}
How we do things to data.
\begin{itemize}
\item types of data for categorizing data; variables for saving data; how
to work with lists of data; now, how to do things to
lists/groupings.
\begin{itemize}
\item also works with strings
\end{itemize}
\item syntax: for item in collection: print(item).
\begin{itemize}
\item practice with both lists and strings
\end{itemize}
\item a note on variable names:
\begin{itemize}
\item the variable following "for" is assigned on the fly
\end{itemize}
\item f-strings
\end{itemize}

String Methods
\begin{itemize}
\item how to do things to strings within loops
\item 'HELLO'.lower()
\item make a list of cities, and make them all lowercasee
\item now save that list to a new list, an empty list
\begin{itemize}
\item why would we want to do this?
\end{itemize}
\end{itemize}

\subsubsection{Group challenge:}
\label{sec:org96532d4}
\begin{itemize}
\item list of prime numbers and their squares, using f strings.
\end{itemize}

\subsection{logic}
\label{sec:orge6c9b26}
Boolean data
\begin{itemize}
\item type() -> True or False
\item evaluates mathematical expressions
\begin{itemize}
\item different operators, look them up. Many different kinds.
\end{itemize}
\item if statement for checking age
\begin{itemize}
\item multiple conditions
\end{itemize}
\end{itemize}

Combining loops with logic:
\begin{itemize}
\item DEFENDING WOMEN FROM GENDER IDEOLOGY EXTREMISM AND RESTORING
BIOLOGICAL TRUTH TO THE FEDERAL GOVERNMENT
\item if it contains the word "gender", "protect", or others, we will
print.
\end{itemize}

\begin{verbatim}
for i in text.split('.'):
    if 'binary' in i:
        print(i)
\end{verbatim}

\subsection{BREAK}
\label{sec:orgdc856a3}

\subsection{csv module}
\label{sec:org95fe6e5}
\begin{itemize}
\item csv module
\begin{itemize}
\item what is a module? a collection of code for doing something, in
this case, for opening csv files
\item read a little of the docs on CSV module, reader
\end{itemize}

\item printing rows from csv on \href{https://data.cityofnewyork.us/City-Government/Enforcement-Actions-Board-Determinations-and-Penal/xrxs-qn95/about\_data}{Campaign Violations}:
\begin{itemize}
\item import csv
\item open the file with open statement
\item print the rows
\end{itemize}
\end{itemize}

\begin{verbatim}
with open('./Downloads/Enforcement_Actions_Board_Determinations_and_Penalties_20250210.csv') as f:
    data = csv.reader(f)
    for row in data:
        print(row)
\end{verbatim}

How do we get just the first object from each column? The names?
\begin{itemize}
\item breakup the problem into parts
\item check the type of data(s)
\item how do we access info from a list?
\end{itemize}

\begin{verbatim}

with open ('violations_sample.csv', 'r') as f:
    data = csv.reader(f)
    for row in data:
        print(row[1])

\end{verbatim}

\subsubsection{group challenge: Combining logic, loops, and csv:}
\label{sec:orgb09298e}
\begin{itemize}
\item search for a specific candidate name in the dataset
\item print out all rows containing that candidate's name
\item advanced: print out only the date and the violation
\end{itemize}


\begin{verbatim}
with open('violations.csv', 'r') as f:
  data = csv.reader(f)
  for row in data:
      if "Eric" in row[1]:
          print(row)
\end{verbatim}

\subsubsection{advanced challenge (that I'll walk you through)}
\label{sec:org9669b28}
Combine what we know from the above with f-strings to write more
complex output.

Write a loop that prints out the Candidate’s name and Violation if
that violation contains the word “contribution” in it. Use f-strings
so that you can format the answer the following:

Name: [candidate name], Violation: [candidate violation]

Here’s the answer, but don’t look until you’ve spent at least 5
mintues working on it!

\begin{verbatim}
with open ('violations.csv', 'r') as f:
  dict_reader = csv.reader(f)
  for row in dict_reader:
      violation = row[3]
      if "contribution" in violation:
          candidate = row[1]
          print(f'Name: {candidate}, Violation: {violation}')
\end{verbatim}


\subsection{next week: web scraping!}
\label{sec:org4836bfd}
\end{document}
