% Created 2025-03-12 Wed 14:39
% Intended LaTeX compiler: pdflatex
\documentclass[11pt]{article}
\usepackage[utf8]{inputenc}
\usepackage[T1]{fontenc}
\usepackage{graphicx}
\usepackage{longtable}
\usepackage{wrapfig}
\usepackage{rotating}
\usepackage[normalem]{ulem}
\usepackage{amsmath}
\usepackage{amssymb}
\usepackage{capt-of}
\usepackage{hyperref}
\author{fcalado}
\date{\today}
\title{}
\hypersetup{
 pdfauthor={fcalado},
 pdftitle={},
 pdfkeywords={},
 pdfsubject={},
 pdfcreator={Emacs 29.3 (Org mode 9.6.15)}, 
 pdflang={English}}
\begin{document}

\tableofcontents

\section{week 7: apis continued}
\label{sec:org8f7ae74}
Goals:
\begin{itemize}
\item review MET API,
\begin{itemize}
\item how to request, parse, and extract data
\item start to do a little data analysis of results
\end{itemize}
\item examine concepts: authentication \& pagination
\begin{itemize}
\item NYT API
\item self guided exploration of NYPL API
\end{itemize}
\item discuss article, start to brainstrom what we want to do
\end{itemize}

\subsection{review: API requests, methods, calls (20m)}
\label{sec:orgf027e9d}
\begin{itemize}
\item discuss the anatomy of an API request
\begin{itemize}
\item root, path, endpoint
\item making the call
\end{itemize}
\item go over some useful methods for working with API data
\begin{itemize}
\item dir(), json(), keys()
\end{itemize}
\item using different paths: review documentation for the MET API: 
\begin{itemize}
\item search path
\item object path
\end{itemize}
\end{itemize}

\begin{verbatim}
base_url = "https://collectionapi.metmuseum.org"
path = "/public/collection/v1/search"
query = "?q=woman"

url = f'{base_url}{path}{query}'
r = requests.get(url)

type(r)
dir(r)

parsed = r.json()

parsed.keys()

\end{verbatim}

\subsection{looping through results to make many calls (20m)}
\label{sec:orgd85b085}
\begin{itemize}
\item we are going to grab a bunch of these, create a dataset of works.
\item let's start with the first fifty, saving them to a list:
\begin{itemize}
\item get the IDs, make a list.
\end{itemize}
\item write a loop that:
\begin{itemize}
\item makes a call to the objects path
\item parses the json
\item saves to a relevant list
\end{itemize}
\end{itemize}

\begin{verbatim}
first_fifty = []
for item in ids:
    # passing the objectID variable into the URL
    base_url = 'https://collectionapi.metmuseum.org'
    path = '/public/collection/v1/objects/'
    url = f'{base_url}{path}{item}'
    # grabbing our response for that object
    response = requests.get(url)
    # parsing our response with json
    parsed = response.json()
    # appending the response to our new list
    first_fifty.append(parsed)
\end{verbatim}

Dealing with errors: two ways: if statements and variables, with get()

\begin{verbatim}
for item in first_fifty:
  if item.get('artistDisplayName'):
    print(item['artistDisplayName'])

for item in first_fiftu: 
  title = item.get('artistDisplayName')
  print(title)
\end{verbatim}

Put it into a dataframe. Make a couple of charts - pie chart, bar
chart, just for demonstration

\begin{verbatim}
# let's get a bunch of this data into lists

titles = []
names = []
genders = []
depts = []
countries = []
urls = []

for item in first_fifty:
    title = item.get('artistGender')
    titles.append(title)
    name = item.get('artistDisplayName')
    names.append(name)
    gender = item.get('artistGender')
    genders.append(gender)
    dept = item.get('department')
    depts.append(dept)
    country = item.get('country')
    countries.append(country)
    url = item.get('objectURL')
    urls.append(url)

import pandas as pd

df = pd.DataFrame({
  'title': titles,
  'name': names,
  'gender': genders,
  'department': depts,
  'country': countries,
  'link': urls
})

df.info()

df.value_counts('department')

df.department.value_counts().plot(kind = 'barh')

df.department.value_counts().plot(kind = 'pie')
\end{verbatim}

\subsection{authentication \& pagination: NYTimes API (40m)}
\label{sec:org1b5f5a7}

Creating API calls with keys
\begin{itemize}
\item requesting an API key
\item adding the key
\end{itemize}

\begin{verbatim}
# key
key = 'HHhPAjnfEF2EmLtf9PdGsLEQZizR3iQu'

# query
query = 'migrant'

# base URL
url = f'https://api.nytimes.com/svc/search/v2/articlesearch.json?&q={query}&api-key={key}'

response = requests.get(url)

response # 200

type(response) # Response

\end{verbatim}

Understanding our structure
\begin{itemize}
\item parse our data
\item we find different structure of data
\begin{itemize}
\item keep going deeper into the data using keys()
\end{itemize}
\end{itemize}

\begin{verbatim}
parsed = response.json()

parsed.keys() # dict_keys(['status', 'copyright', 'response'])

parsed['response'].keys() # dict_keys(['docs', 'meta'])

type(parsed['response']['meta']) # dict

parsed['response']['meta']

type(parsed['response']['docs']) # list

# The very first item in `docs` appears to be information about a
# single article.

parsed['response']['docs'][0]

parsed['response']['docs'][0].keys()

# If I wanted to proceed to get this data, I would write a loop and
# assign variables
\end{verbatim}

Paginating through results
\begin{itemize}
\item We have 10 articles, because the NYTimes API only allows us to get
10 results at a time.
\item We can paginate through the results by adding an \&page= parameter,
combined with a sleep() function, which allows us to insert pauses
in our request (as to not overload the NYTimes servers).
\item write a loop that goes through a range() of 5 pages, and does a
search for each page.
\item append the results for each page to a list called results
\item check out our results, where is our data?
\end{itemize}

\begin{verbatim}
results = []
query = 'migrant'

for i in range(0, 5):  
    url = f'https://api.nytimes.com/svc/search/v2/articlesearch.json?q={query}&page={i}&api-key={key}'
    response = requests.get(url)
    parsed = response.json()
    articles = parsed['response']['docs']
    results.append(articles)
    sleep(6) # sleep at least 6 seconds not to overload the servers
\end{verbatim}

In the end, we have a list of results, 5 lists to be exact (each list
represents one page of articles). Each result (or page) contains
information about 10 articles that matched our search.

\begin{verbatim}

results[0][0].keys() # for the very first item on the very first page. 

results[0][0]['abstract']

\end{verbatim}


BREAK
\subsection{group challenge: the NYPL API}
\label{sec:org700b823}
Walk them through the authentication process. 

Challenge: use the search path, and figure out a way to get more
information about the individual objects in the search path results.

\subsection{discuss artivism article}
\label{sec:orgfab6cf4}

What is "data artivism"?
\begin{itemize}
\item works that mobilize art and craft as tools of social contestation
and (de)construction and as methods to engage with and visualize
data
\item not just about making things accessible that weren't before, not
about representation through presence. But about changing \emph{how} we
see things.
\end{itemize}

Buidling on concept of "counterdata"

\begin{quote}
The production of “counterdata” (D’Ignazio and Klein, 2020) on
feminicide—a careful practice that involves researching, recording,
and remembering the lives of women lost to violence (D’Ignazio et al.,
2022)—has been a key practice for activists to visibilizar the issue.
With their data, activists hope to move society from silence and
indifference to outrage and action (Suárez Val, 2021). They also aim
to contest the hegemonic visuality of feminicide: “an aesthetic order
[…] made up of structures of visibility and invisibility that classify
and separate individuals into grievable, political lives and
culturally objectified lives” (Blaagaard, 2019: 253) that is dominated
by often glamorized mainstream representations of violence against
women, such as crime shows or news coverage.
\end{quote}

The point of amassing this data is not only to make what is not often
visualized more visible, but to also change the kind of visiblity that
a topic usually gets: with gender-based violence, we have a lot of
gratuitous representations in movies, the battered woman, for example,
we also have crime shows.

There's an example of a visualization project, "No estamos todas",
that doesn't just enumerate the number of deaths, but to remember them
by their beauty. They "create an illustration for each woman to
remember her in life, not as another crime statistic."

A possibility for your project, thinking about using your project for
taking data and maybe presenting it in another way. 

This kind of "data artivism" wants to represent data, differently to
move people to action instead of just spectating.
\end{document}
